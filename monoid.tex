% Created 2018-04-03 Tue 17:05
% Intended LaTeX compiler: pdflatex
\documentclass[presetation]{beamer}
\usepackage[utf8]{inputenc}
\usepackage[T1]{fontenc}
\usepackage{graphicx}
\usepackage{grffile}
\usepackage{longtable}
\usepackage{wrapfig}
\usepackage{rotating}
\usepackage[normalem]{ulem}
\usepackage{amsmath}
\usepackage{textcomp}
\usepackage{amssymb}
\usepackage{capt-of}
\usepackage{hyperref}
\usetheme{Madrid}
\author{Evan Misshula}
\date{2018-03-28}
\title{Why teaching functional programming to undergraduates at CUNY is important}
\hypersetup{
 pdfauthor={Evan Misshula},
 pdftitle={Why teaching functional programming to undergraduates at CUNY is important},
 pdfkeywords={},
 pdfsubject={},
 pdfcreator={Emacs 25.2.1 (Org mode 9.0.6)}, 
 pdflang={English}}
\begin{document}

\maketitle





\section{Monoids}
\label{sec:org860c089}
\begin{frame}[fragile,label={sec:orga6509a5}]{Monoid Definition}
 \begin{definition}[Monoid definition]
A data type, category or set is a \alert{monoid} if it has a binary
operation \textbullet{} which is associative and has an identity.
\begin{itemize}
\item \(\forall a,b,c \in S, (a \bullet b) \bullet c = a \bullet (b \bullet c)\)
\item \(e \bullet a = a \bullet e = a\)
\end{itemize}

\begin{verbatim}
:set +m
:{
class Monoid m where
    mempty :: m
    mappend :: m -> m -> m
    mconcat :: [m] -> m
    mconcat = foldr mappend mempty
:}
\end{verbatim}
\end{definition}
\end{frame}

\begin{frame}[label={sec:org65f3680}]{Monoid functions defined}
\begin{block}{Defining the monoid functions}
\begin{itemize}
\item 'mempty' is just the identity function
\item mappend is the binary function
\begin{itemize}
\item \alert{it doesn't just append}
\end{itemize}
\item mconcat reduces a list of monoid values and reduces them to one by
applying mappend
\end{itemize}
\end{block}
\end{frame}

\begin{frame}[label={sec:org8ab1b3a}]{Monoid Laws}
\begin{theorem}[The Monoid Laws are just the definition in Haskell]
\begin{itemize}
\item mappend mempty x = x
\item mappend x mempty = x
\item mappend (mappend x y) z = mappend x (mappend y z)
\end{itemize}
\end{theorem}
\end{frame}
\section{Lists are monoids}
\label{sec:org32c4724}
\begin{frame}[label={sec:org173991c}]{Monoid examples}
\begin{example}[List is a monoid]
\begin{itemize}
\item\relax [] with (++) is a monoid
\begin{itemize}
\item id = ""
\end{itemize}
\item Natural numbers with (*) is a monoid
\begin{itemize}
\item id = 1
\end{itemize}
\item Natural numbers with (+) is a monoid
\begin{itemize}
\item id = 0
\end{itemize}
\end{itemize}
\end{example}
\end{frame}
\end{document}
% Created 2018-04-03 Tue 16:31
% Intended LaTeX compiler: pdflatex
\documentclass[presetation]{beamer}
\usepackage[utf8]{inputenc}
\usepackage[T1]{fontenc}
\usepackage{graphicx}
\usepackage{grffile}
\usepackage{longtable}
\usepackage{wrapfig}
\usepackage{rotating}
\usepackage[normalem]{ulem}
\usepackage{amsmath}
\usepackage{textcomp}
\usepackage{amssymb}
\usepackage{capt-of}
\usepackage{hyperref}
\usetheme{Madrid}
\author{Evan Misshula}
\date{2018-03-28}
\title{Why teaching functional programming to undergraduates at CUNY is important}
\hypersetup{
 pdfauthor={Evan Misshula},
 pdftitle={Why teaching functional programming to undergraduates at CUNY is important},
 pdfkeywords={},
 pdfsubject={},
 pdfcreator={Emacs 25.2.1 (Org mode 9.0.6)}, 
 pdflang={English}}
\begin{document}

\maketitle

\section{The newtype keyword}
\label{sec:org8d1fc5a}
\begin{frame}[label={sec:org620bd4c}]{Some lessons we've skipped}
\begin{block}{Defining types}
\begin{itemize}
\item \emph{data} will define a new algebraic type
\item \emph{type} creates a type synonym
\item \emph{newtype} creates new types from old types
\end{itemize}
\end{block}
\end{frame}

\begin{frame}[fragile,label={sec:org6f749a0}]{Applicative Functor in two ways}
 \begin{columns}
\begin{column}{.48\columnwidth}
\begin{block}{function left, each argument right}
\begin{verbatim}
:m Control.Applicative
[(+1),(*100),(*5)] <*> [1..3]
1==1
\end{verbatim}

\begin{verbatim}

[2,3,4,100,200,300,5,10,15]
\end{verbatim}
\end{block}
\end{column}

\begin{column}{.48\columnwidth}
\begin{block}{function left, every argument right}
\begin{verbatim}
:set +m
:{
instance Applicative ZipList where  
	pure x = ZipList (repeat x)  
	ZipList fs <*> ZipList xs = ZipList (zipWith (\f x -> f x) fs xs)  
:}
  getZipList $ ZipList [(+1),(*100),(*5)] <*> ZipList [1,2,3]
  -- getZipList $
  -- ZipList [(+1),(*100),(*5)]
  --  <*> ZipList [1,2,3]
1==1
\end{verbatim}

\begin{verbatim}

Prelude Control.Applicative| Prelude Control.Applicative| Prelude Control.Applicative| Prelude Control.Applicative| Prelude Control.Applicative> [2,200,15]
\end{verbatim}
\end{block}
\end{column}
\end{columns}
\end{frame}



\begin{frame}[label={sec:org398e7f7}]{The newtype keyword}
\begin{block}{'newtype' takes one type and wrap it}
\begin{itemize}
\item to present it as another type
\item[{\textasciitilde{}newtype ZipList a = ZipList \{getZipList}] [a]\}\textasciitilde{}
\item data can have multiple value contstructors
\end{itemize}
\end{block}
\end{frame}
\begin{frame}[fragile,label={sec:org30d08f6}]{type vs. newtype vs. data examples}
 \begin{block}{'data' to make new types}
\begin{itemize}
\item Here are additive and multiplicative types with multiple constructors
\end{itemize}
\begin{verbatim}
data Profession = Fighter | Archer | Wizard
data Species = Human | Elf | Orc | Goblin
data PlayerCharacter = PlayerCharacter Species Profession
\end{verbatim}
\end{block}
\end{frame}

\begin{frame}[fragile,label={sec:orge6b2485}]{Using newtype to drive typeclass properties}
 \begin{block}{newtype}
\begin{verbatim}
newtype CharList = CharList {getCharList :: [Char]} deriving(Eq,Show)
CharList "this will be shown!"
CharList "benny" == CharList "benny"
CharList "benny" == CharList "oisters"
1==1
\end{verbatim}

\begin{verbatim}

CharList {getCharList = "this will be shown!"}
True
False
\end{verbatim}
\end{block}
\end{frame}
\end{document}
% Created 2018-04-03 Tue 15:53
% Intended LaTeX compiler: pdflatex
\documentclass[presetation]{beamer}
\usepackage[utf8]{inputenc}
\usepackage[T1]{fontenc}
\usepackage{graphicx}
\usepackage{grffile}
\usepackage{longtable}
\usepackage{wrapfig}
\usepackage{rotating}
\usepackage[normalem]{ulem}
\usepackage{amsmath}
\usepackage{textcomp}
\usepackage{amssymb}
\usepackage{capt-of}
\usepackage{hyperref}
\usetheme{Madrid}
\author{Evan Misshula}
\date{2018-03-28}
\title{Why teaching functional programming to undergraduates at CUNY is important}
\hypersetup{
 pdfauthor={Evan Misshula},
 pdftitle={Why teaching functional programming to undergraduates at CUNY is important},
 pdfkeywords={},
 pdfsubject={},
 pdfcreator={Emacs 25.2.1 (Org mode 9.0.6)}, 
 pdflang={English}}
\begin{document}

\maketitle

\section{Applicative functors}
\label{sec:org2513024}
\begin{frame}[fragile,label={sec:org7500da3}]{What if we map a multi-parameter function over a functor?}
 \begin{itemize}
\item Look at the type signature
\end{itemize}
\begin{verbatim}
a = fmap (*) [1..4]
:t a
fmap (\f -> f 9) a
1==1
\end{verbatim}

\begin{verbatim}

a :: (Num a, Enum a) => [a -> a]
[9,18,27,36]
\end{verbatim}
\end{frame}

\begin{frame}[fragile,label={sec:orgdcfb95f}]{What if we want to take a function out of a Just}
 \begin{block}{Let's take a Just (3 *) and map}
and map it over Just 5
\begin{verbatim}
:set +m
:{
class (Functor f) => Applicative f where
    pure :: a -> f a;
    (<*>) :: f (a -> b) -> f a -> f b
:}
\end{verbatim}
\end{block}
\end{frame}

\begin{frame}[fragile,label={sec:org8e7ac2e}]{Maybe Applicative}
 \begin{block}{Let's look at the Applicative for Maybe}
\begin{verbatim}
:set +m
:{
instance Applicative MyMaybe where
    pure = Just
    Nothing <*> _ = Nothing
    (Just f) <*> something = fmap f something
:}
\end{verbatim}
\end{block}
\end{frame}

\begin{frame}[fragile,label={sec:org12d88ee}]{Maybe Applicative inside the repl}
 \begin{block}{Using the Maybe Applicative}
\begin{verbatim}
-- :add Control.Applicative
Just (+3) <*> Just 9
pure (*2) <*> Just 10
pure (+3) <*> Just 9
Just (++"!!") <*> Just "Go now"
Nothing <*> Just "woot"
1==1
\end{verbatim}

\begin{verbatim}

<interactive>:587:1: error:
    • Could not deduce (Applicative Maybe) arising from a use of ‘<*>’
      from the context: Num b
        bound by the inferred type of it :: Num b => Maybe b
        at <interactive>:587:1-20
    • In the expression: Just (+ 3) <*> Just 9
      In an equation for ‘it’: it = Just (+ 3) <*> Just 9
<interactive>:588:1: error:
    • Could not deduce (Applicative Maybe) arising from a use of ‘<*>’
      from the context: Num b
        bound by the inferred type of it :: Num b => Maybe b
        at <interactive>:588:1-21
    • In the expression: pure (* 2) <*> Just 10
      In an equation for ‘it’: it = pure (* 2) <*> Just 10
<interactive>:589:1: error:
    • Could not deduce (Applicative Maybe) arising from a use of ‘<*>’
      from the context: Num b
        bound by the inferred type of it :: Num b => Maybe b
        at <interactive>:589:1-20
    • In the expression: pure (+ 3) <*> Just 9
      In an equation for ‘it’: it = pure (+ 3) <*> Just 9
<interactive>:590:1: error:
    • No instance for (Applicative Maybe) arising from a use of ‘<*>’
    • In the expression: Just (++ "!!") <*> Just "Go now"
      In an equation for ‘it’: it = Just (++ "!!") <*> Just "Go now"
<interactive>:591:1: error:
    • No instance for (Applicative Maybe) arising from a use of ‘<*>’
    • In the expression: Nothing <*> Just "woot"
      In an equation for ‘it’: it = Nothing <*> Just "woot"
\end{verbatim}
\end{block}
\end{frame}


\begin{frame}[fragile,label={sec:orgdc1d982}]{Fmap as an infix operator}
 \begin{block}{Control.Applicative exports a function called <\$>}
which is fmap as an infix operator
\begin{verbatim}
(<$>) :: (Functor f) => (a->b) -> f a -> f b
f <$> x = fmap f x
\end{verbatim}
\end{block}
\end{frame}

\begin{frame}[fragile,label={sec:org76bae1d}]{Compare Applicatives in the repl}
 \begin{block}{Infix fmap in the repl}
\begin{verbatim}
(++) <$> Just "John " <*> Just "Travolta"
(++) "John " "Travolta"
1==1 
\end{verbatim}

\begin{verbatim}
<interactive>:597:1: error:
    • No instance for (Applicative Maybe) arising from a use of ‘<*>’
    • In the expression: (++) <$> Just "John " <*> Just "Travolta"
      In an equation for ‘it’:
          it = (++) <$> Just "John " <*> Just "Travolta"
John Travolta
\end{verbatim}
\end{block}
\end{frame}

\begin{frame}[fragile,label={sec:orgae70591}]{Lists are Applicative Functors}
 \begin{definition}[Definition of the Applicative for a list]
\begin{itemize}
\item Literally a Cartesian product of functions and list values
\end{itemize}
\begin{verbatim}
:set +m
:{
instance Applicative [] where
    pure x = [x]
    fs <*> xs = [f x | f <- fs, x<- xs]
:}
\end{verbatim}
\end{definition}
\end{frame}

\begin{frame}[fragile,label={sec:org5c2dcab}]{Applicative Functors of lists in the repl}
 \begin{block}{Applicative Functors of lists in the repl}
\begin{verbatim}
[(*0),(+100),(^2)] <*> [1..4]
[(+),(*)] <*>[1,2]<*>[3,4]
(++) <$> ["ha","heh","hmm"] <*> ["?","!","."]
1==1
\end{verbatim}

\begin{verbatim}
[0,0,0,0,101,102,103,104,1,4,9,16]
[4,5,5,6,3,4,6,8]
["ha?","ha!","ha.","heh?","heh!","heh.","hmm?","hmm!","hmm."]
\end{verbatim}
\end{block}
\end{frame}

\begin{frame}[fragile,label={sec:org2b76b7f}]{IO is an Applicative}
 \begin{block}{Let's see how the IO Applicative is implemented:}
\begin{verbatim}
:set +m
:{
instance Applicative IO where
    pure = return
    a <*> b = do
	f <- a
	x <- b
	return (f x)
:}
\end{verbatim}
\end{block}
\end{frame}

\begin{frame}[fragile,label={sec:orgb8e8759}]{Concatenating IO strings}
 \begin{columns}
\begin{column}{.48\columnwidth}
\begin{block}{Two ways to concatenate two lines of user input string}
\begin{itemize}
\item Imperative code
\end{itemize}
\begin{verbatim}
:set +m
:{
myAction :: IO String
myAction = do
    a <- getLine
    b <- getLine
    return $ a ++ b
:}
\end{verbatim}
\end{block}
\end{column}

\begin{column}{.48\columnwidth}
\begin{block}{Applicative way to concatenate two lines of user input string}
\begin{itemize}
\item Applicative code
\end{itemize}
\begin{verbatim}
:set +m
:{
myAction :: IO String
myAction = (++) 
	    <$> getLine 
	    <*> getLine
:}
\end{verbatim}
\end{block}
\end{column}
\end{columns}
\end{frame}

\begin{frame}[label={sec:org725cf53}]{The first Applicative Functor Law}
\begin{theorem}[The first Applicative Functor Law]
\begin{itemize}
\item pure f <*> x = fmap f x
\end{itemize}
\end{theorem}
\end{frame}
\end{document}
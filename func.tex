% Created 2018-04-03 Tue 13:07
% Intended LaTeX compiler: pdflatex
\documentclass[presetation]{beamer}
\usepackage[utf8]{inputenc}
\usepackage[T1]{fontenc}
\usepackage{graphicx}
\usepackage{grffile}
\usepackage{longtable}
\usepackage{wrapfig}
\usepackage{rotating}
\usepackage[normalem]{ulem}
\usepackage{amsmath}
\usepackage{textcomp}
\usepackage{amssymb}
\usepackage{capt-of}
\usepackage{hyperref}
\usetheme{Madrid}
\author{Evan Misshula}
\date{2018-03-28}
\title{Why teaching functional programming to undergraduates at CUNY is important}
\hypersetup{
 pdfauthor={Evan Misshula},
 pdftitle={Why teaching functional programming to undergraduates at CUNY is important},
 pdfkeywords={},
 pdfsubject={},
 pdfcreator={Emacs 25.2.1 (Org mode 9.0.6)}, 
 pdflang={English}}
\begin{document}

\maketitle

\section{Functor Laws}
\label{sec:orgeaca1d8}
\begin{frame}[label={sec:orgeb2cca6}]{Functor Law intuition}
\begin{block}{If functors mean that something can be mapped over\ldots{}}
\begin{itemize}
\item then calling 'fmap' on a functor should
\begin{itemize}
\item map a function over the functor
\end{itemize}
\end{itemize}
\pause
\begin{itemize}
\item \alert{Nothng else}
\end{itemize}
\end{block}
\end{frame}

\begin{frame}[label={sec:org8feac8d}]{The First Functor Laws}
\begin{definition}[The First Functor Law]
states that if we map the identity (id) function over a functor, we
get the functor
\begin{itemize}
\item fmap id = id
\end{itemize}
\end{definition}
\end{frame}

\begin{frame}[fragile,label={sec:org2f384cf}]{Identity in the Repl}
 \begin{block}{Identity functions in the repl}
\begin{verbatim}
fmap id (Just 3)
id (Just 3)
fmap id [1..5]
id [1..5]
fmap id []
fmap id Nothing
1==1
\end{verbatim}

\begin{verbatim}
Just 3
Just 3
[1,2,3,4,5]
[1,2,3,4,5]
[]
Nothing
\end{verbatim}
\end{block}
\end{frame}

\begin{frame}[label={sec:org6bdd93c}]{The Second Functor Law}
\begin{definition}[The Second Functor Law says]
The Second Functor Law says that composing two functions and then
mapping the composed function over a functor is the same as first
mapping one function over the functor and then mapping the other one.
\begin{itemize}
\item fmap (f.g) = fmap f . fmap g
\item fmap (f.g) F = fmap f (fmap g F)
\end{itemize}
\end{definition}
\end{frame}

\begin{frame}[fragile,label={sec:org7e87142}]{Composition in the Repl}
 \begin{block}{Composition functions in the repl}
\begin{verbatim}
fmap ((+1).(*2)) (Just 3)
fmap (+1) (fmap (*2) (Just 3))
fmap  ((+1).(*2)) [1..5]
fmap (+1) (fmap (*2) [1..5])
1==1
\end{verbatim}

\begin{verbatim}
Just 7
Just 7
[3,5,7,9,11]
[3,5,7,9,11]
\end{verbatim}
\end{block}
\end{frame}
\end{document}